\chapter{Fazit \& Ausblick}
\label{ch:conclusions}

Diese Arbeit zeigt auf, dass Machine Learning durchaus als Kreativitätswerkzeug verwendet werden kann und ein
neuronales Netz fähig ist
Die Qualität und der Umfang der Erzeugnisse stehen und fallen allerdings mit der Qualität des Trainingsdatensatzes.
Hier könnte mit einem Daten-Preprocessing abgeholfen werden, indem beispielsweise die Namen der Gerichte nach
Sprache segmentiert werden.
So könnte auch eine weitere Klassifikation nach Sprache stattfinden und die Variabilität des Modells erhöhen, indem
die Sprache als Steuerparameter verwendet wird.
Es hat sich gezeigt, dass die Klassifikation nach Zeit (Jahr) nicht funktioniert, weil die Verteilung der Gerichte nach Jahr sehr unregelmässig ist und
weil sich die Epoche nicht sonderlich in der Wahl von Wörtern niederschlägt.

Das Modell könnte so weiterentwickelt werden, dass Feedback vom Benutzer integriert werden kann und so die Qualität des Modells nach und nach verbessert wird.
Beispielsweise könnte der Datensatz manuell um weitere Labels erweitert werden.
Zum einen wäre sicher die Sprache der Namen interessant, zum anderen auch mehr Informationen zum Gericht selbst.
Ist ein Gericht vegetarisch oder gar vegan?
Wieviele Kalorien enthält ein Gericht?
Ist es eher deftig oder leicht?
Nicht zuletzt ist ja die Normalisierung des Preises Gegenstand von tatsächlichen Weiterentwicklungen des Datensatzes\footnote{http://nypl.github.io/menus-api/, Stichwort «price normalization»}.
Ist dieser Schritt vollendet, so wäre es möglich, günstige und teure Gerichte zu trainieren.
All diese Daten eignen sich wahrscheinlich besser als die Zuordnung einer Epoche, da die Daten deutlicher zugeordnet werden könnten.
So haben Gerichte ohne Fleisch die Tendenz, günstiger auszufallen während Angus Rind erfahrungsgemäss teuer ist.

Eine weitere Arbeit könnte darin bestehen, die optimale Netzkonfiguration für den höchsten Anteil lesbarer Wörter sowie die
höchste Originalität zu finden.
Möglicherweise wäre auch die Veränderung der Sequenz-Fenstergrösse interessant, um das Modell leistungsfähiger zu machen.
