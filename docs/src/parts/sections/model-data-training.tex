\section{Daten \& Training}
\label{sec:model-data-training}

In diesem Abschnitt werden die Trainingsdaten für das Modell analysiert und beschrieben.
Als Lerngrundlage für den Algorithmus dienen rund 400000 Namen von Gerichten.
Diese sind im Kaggle-Datensatz «What's on the menu»\footnote{https://www.kaggle.com/nypl/whats-on-the-menu}\footnote{http://nypl.github.io/menus-api/} enthalten.
Der Unterdatensatz «Dish.csv» enthält folgende Struktur:


\begin{center}
    \resizebox{\textwidth}{!}{
    \begin{tabular}{ |r|l|r|r|r|r|r|r| }
        \hline
        \textbf{id} & \textbf{name} & \textbf{menus\_appeared} & \textbf{times\_appeared} & \textbf{first\_appeared} & \textbf{last\_appeared} & \textbf{lowest\_price} & \textbf{highest\_price} \\
        \hline
        1 & Consomme printaniere royal  &    8 &    9 & 1897 & 1927 & 0.2  & 0.4 \\
        \hline
        2 & Chicken gumbo               &  110 &  116 & 1895 & 1960 & 0.1  & 0.8 \\
        \hline
        3 & Tomato aux croutons         &   13 &   13 & 1893 & 1917 & 0.25 & 0.4 \\
        \hline
        4 & Onion au gratin             &   41 &   41 & 1900 & 1971 & 0.25 & 1   \\
        \hline
        5 & Radishes                    & 3265 & 3349 & 1854 & 2928 & 0    & 25  \\
        \hline
        6 & Chicken soup with rice      &   48 &   49 & 1897 & 1961 & 0.1  & 0.6 \\
        \hline
        … & … & … & … & … & … & … & … \\
        \hline
    \end{tabular}
    }
\end{center}

Für die Arbeit relevant ist vorerst lediglich die \textit{name}-Spalte.
Sie enthält Bezeichungen von Gerichten in mehreren Sprachen.

Für das Training eines ersten Modells wird folgende Spezifikation verwendet:

Batches: 100
Batchsize: 50
Epochen: 50
RNN-Units: 64
RNN-Layers: 1
Textlänge: 15

Für die Validierung pro Epoche werden jeweils die weiteren Zeilen, die auf das Trainingsset folgen, verwendet.


