\section{Daten}
\label{sec:model-data-training}

Als Lerngrundlage für den Algorithmus dienen rund 400'000 Bezeichnungen von Gerichten.
Diese sind im Datensatz «What's on the menu»\footnote{http://nypl.github.io/menus-api/} enthalten, der von
der New York Public Library unterhalten und laufend erweitert wird.
Der Unterdatensatz «Dish.csv» besteht aus folgender Struktur:

\begin{center}
    \resizebox{\textwidth}{!}{
    \begin{tabular}{ |r|l|r|r|r|r|r|r| }
        \hline
        \textbf{id} & \textbf{name} & \textbf{menus\_appeared} & \textbf{times\_appeared} & \textbf{first\_appeared} & \textbf{last\_appeared} & \textbf{lowest\_price} & \textbf{highest\_price} \\
        \hline
        1 & Consomme printaniere royal  &    8 &    9 & 1897 & 1927 & 0.2  & 0.4 \\
        \hline
        2 & Chicken gumbo               &  110 &  116 & 1895 & 1960 & 0.1  & 0.8 \\
        \hline
        3 & Tomato aux croutons         &   13 &   13 & 1893 & 1917 & 0.25 & 0.4 \\
        \hline
        4 & Onion au gratin             &   41 &   41 & 1900 & 1971 & 0.25 & 1   \\
        \hline
        5 & Radishes                    & 3265 & 3349 & 1854 & 2928 & 0    & 25  \\
        \hline
        6 & Chicken soup with rice      &   48 &   49 & 1897 & 1961 & 0.1  & 0.6 \\
        \hline
        … & … & … & … & … & … & … & … \\
        \hline
    \end{tabular}
    }
\end{center}

Für die Arbeit relevant sind vorerst lediglich die \textit{name}-Spalte sowie die \textit*{firs\_appeared}-Spalte.
Die Namens-Spalte enthält Bezeichnungen in verschiedenen Sprachen, hauptsächlich jedoch in Englisch.
First\_appeared datiert den Zeitpunkt, an dem ein bestimmtes Gericht zum ersten Mal registriert wurde.
Leider sind die Preis-Spalten nicht normalisiert, d.h. Inflation und Lohnniveau sind nicht berücksichtigt.
Somit können Preise unterschiedlicher Epochen nicht verglichen werden bzw. ein Gericht kann nicht per se als «teuer» oder
«günstig» klassifiziert werden.
Möglich wäre die Preise anhand einer eigenen einfachen Funktion in Abhängigkeit der Jahreszahl linear zu normalisieren, was
den Rahmen der Arbeit jedoch sprengen würde.
