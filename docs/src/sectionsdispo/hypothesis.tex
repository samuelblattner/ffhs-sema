\section*{Forschungsfrage}
\label{sec:hypotehsis}

Nachdem die oben genannten Beispiele für Unterhaltung sorgen, stellt sich aber auch die Frage nach deren Nutzen.
Im Zusammenhang mit der in der Einleitung formulierten Motivation, Gastronomiebetriebe in ihrer Kreativität zu unterstützen, soll in
dieser Arbeit folgender Fragestellung nachgegangen werden:

\textbf{\guillemotleft Kann ein einfaches Character-based-RNN als Kreativitätswerkzeug für Gastronomiebetriebe dienen, indem es
neue Vorschläge für Bezeichnungen von Gerichten liefert?\guillemotright{}}

Im Weiteren muss die Frage geklärt werden, \textit{wie} die erzeugten Vorschläge als Inspiration funktionieren können.
Mit der Bezeichnung allein ist noch keine neue Speise erschaffen.
Jedoch können neuartige Begriffe oder ungewohnte Wortkombinationen weitere Gedanken und Ideen z.B. für Ingredienzen oder Geschmacksrichtungen auslösen.


