\section*{Einleitung}
\label{sec:introduction}

Die Gastronomie hat sich im vergangenen Jahrzehnt zu einem äusserst populären Thema entwickelt.
Haupttreiber dafür sind sicherlich Social Media, auf denen kreative Lokale, ausgefallene Gerichte und besondere
Momente weltweit geteilt werden.
Andererseits tragen auch ein neues Ernährungsbewusstsein, die weltweite Mobilität sowie Berichterstattungen über Gastronomiebetriebe, beispielsweise als TV-Serie, zum wachsenden Interesse bei.
Der Markt ist hart umkämpft, und Restaurants kommen und gehen insbesondere in Stadtgebieten in hohen Raten, wie eine neuere
Statistik der Stadt Zürich zeigt\footnote{https://www.zkb.ch/media/pub/coporate/zuerich-in-zahlen/kt-zh-zahlen-223653.pdf}.
Der gesättigte Markt zwingt die Betriebe zu Effizienz, Innovation und möglichst hoher Qualität.
Unlängst hat deshalb auch die Digitalisierung in Gastronomiebetrieben einzug gehalten, werden doch zumindest Reservationen immer
häufiger online getätigt\footnote{https://www.lunchgate.info/blog/gastronomie-digitalisierung}, Bestellungen via Tablet
direkt in die Küche gesendet und Rechnungen per App unter Freunden aufgeteilt.
All diese Entwicklungen beschäftigen sich in erster Linie mit der Automatisierung von Abläufen.
Wie aber können Gastronomiebetriebe in anderen Bereichen, wie z.B. der Kreativität unterstützt werden?


