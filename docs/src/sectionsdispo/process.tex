\section*{Vorgehen}
\label{sec:process}

Zur Beantwortung der Forschungsfrage wird auf Basis des \guillemotleft What's on the menu\guillemotright{}-Datensatzes von Kaggle\footnote{https://www.kaggle.com/nypl/whats-on-the-menu} ein einfaches RNN trainiert, mit dem Ziel, anschliessend
neuartige Bezeichungen von Gerichten erzeugen zu können, die einem Koch oder Restaurantbetreiber als Inspiration für neue Kreationen bzw. Zutatenkombinationen dienen.
Das Modell wird bewusst möglichst einfach gehalten und auf einer vorhandenen GPU trainiert.
Cloud-Lösungen zum Training künstlicher Intelligenz werden in dieser Arbeit nicht verwendet, da einerseits die Erfahrung damit nicht vorhanden ist und andererseits der zeitliche Rahmen nicht ausreicht.

Der Zeitraum für die Bearbeitung der Seminararbeit dauert von 25. Februar (1. Präsenz) bis zum 17. Juni 2019 (5. Präsenz, Präsentation).
Dies entspricht 16 Kalenderwochen.
Es wird davon ausgegangen, dass pro Kalenderwoche 4 Stunden für die Seminararbeit eingesetzt werden können.
Für die Verteilung der Arbeitsblöcke wird die meiste Zeit für das Experientieren, die Aneignung von Wissen und die Ideenfindung aufgewendet.

\paragraph{\textbf{Idee, Konzeption, Literaturrecherche \& Disposition: 25. Februar - 3. Mai 2019}}
\begin{itemize}
    \item Recherche, Experimente, Formulieren der Idee und der Forschungsfrage für die Seminararbeit
    \item Recherche und Zusammentragen relevanter Literatur
    \item Erstellen der Disposition
\end{itemize}

Danach gliedert sich die verbleibende Zeit im Mai und Juni in jeweils zweiwöchige Abschnitte wie folgt:

\paragraph{Modell \& Training: 6. Mai - 17. Mai 2019, 8h}
\begin{itemize}
    \item Erstellen und Trainieren eines einfachen Character-based RNN auf Basis des \guillemotleft What's on the menu\guillemotright{}-Datensatzes von Kaggle
\end{itemize}

\paragraph{Texterzeugung \& Auswertung: 20. Mai - 31. Mai 2019, 24h}
\begin{itemize}
    \item Zusammentragen der Resultate
    \item Beurteilen von Resultaten und Verbesserungsmöglichkeiten
    \item Ggf. Verbesserungen am Modell, Ergründen weiterer Einsatzzwecke
\end{itemize}

\paragraph{Formatierung, Kontrolle, Erstellen Präsentation: 3. Juni - 17. Juni 2019, 12h}
\begin{itemize}
    \item Kontrolle und Verbessern des Inhalts
    \item Erstellen der Präsentation
\end{itemize}